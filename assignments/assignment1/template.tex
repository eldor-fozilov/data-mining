\documentclass{article}
\usepackage[utf8]{inputenc}
\usepackage{graphicx}
\usepackage{hyperref}       % hyperlinks
\usepackage{url}            % simple URL typesetting
\usepackage{booktabs}       % professional-quality tables
\usepackage{amsfonts}       % blackboard math symbols
\usepackage{nicefrac}       % compact symbols for 1/2, etc.
\usepackage{mathtools}
\usepackage{amsmath}


%%%%% Please play around with this template. Try to understand the difference between inline text mode and equation mode. You will see that there are dollar ($) signs in the text. This allows you to write
%%%%% in math mode. If you want to write an equation, you can do as the one below, but there are also other ways. Try google it!

\title{Assignment 1}
\date{March 2022}
\author{%
  HongGilDong \\ % You should type your name here
  20291291\\% Your student number
  UNIST\\
  \texttt{dmlabisthebest@unist.ac.kr} \\% You email
}
\begin{document} % Your document starts here


\maketitle % This creates the title 

\section{K-means algorithm} % You can create section
The k-means algorithm aims to divide a given set of observations into a user-defined number of $k$ clusters. All observations $x$ are allocated to their nearest center-point during each update step (see equation \eqref{eqn:kmeans_assign_step}). % You can reference the equation number by using the command \eqref. This creates a hyperlink when you click the number. \\

% You can write equations like below.
\begin{equation}
S_i^{(t)} = \big \{ x_p : \big \| x_p - \mu^{(t)}_i \big \|^2 \le \big \| x_p - \mu^{(t)}_j \big \|^2 \ \forall j, 1 \le j \le k \big\}
\label{eqn:kmeans_assign_step}
\end{equation}

% Now, you can start your homework from here. Make sure you read the instruction before you start! Good luck with your semester!

\end{document}
