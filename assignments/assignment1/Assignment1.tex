\documentclass{article}
\usepackage[utf8]{inputenc}
\usepackage{graphicx}
\usepackage{hyperref}       % hyperlinks
\usepackage{url}            % simple URL typesetting
\usepackage{booktabs}       % professional-quality tables
\usepackage{amsfonts}       % blackboard math symbols
\usepackage{nicefrac}       % compact symbols for 1/2, etc.
\usepackage{mathtools}
\usepackage{amsmath}

\usepackage{bm}

%%%%% Please play around with this template. Try to understand the difference between inline text mode and equation mode. You will see that there are dollar ($) signs in the text. This allows you to write
%%%%% in math mode. If you want to write an equation, you can do as the one below, but there are also other ways. Try google it!

\title{Assignment 1}
\date{March 2022}
\author{%
  Eldor Fozilov \\ % You should type your name here
  20192032 \\ % Your student number
  UNIST \\
  \texttt{eldorfozilov@unist.ac.kr} \\% You email
}
\begin{document} % Your document starts here


\maketitle % This creates the title 

\section{K-means algorithm} % You can create section
The k-means algorithm aims to divide a given set of observations into a user-defined number of $k$ clusters. All observations $x$ are allocated to their nearest center-point during each update step (see equation \eqref{eqn:kmeans_assign_step}).\\  % You can reference the equation number by using the command \eqref. This creates a hyperlink when you click the number. \\

% You can write equations like below.
\begin{equation}
S_i^{(t)} = \big \{ x_p : \big \| x_p - \mu^{(t)}_i \big \|^2 \le \big \| x_p - \mu^{(t)}_j \big \|^2 \ \forall j, 1 \le j \le k \big\}
\label{eqn:kmeans_assign_step}
\end{equation}

% Now, you can start your homework from here. Make sure you read the instruction before you start! Good luck with your semester!

\section{Maximum Likelihood Estimate [10pt]}
The likelihood function is nothing but a parameterized density $p(D\,|\,\bm{\theta}\ )$ that is used to model a set of data $D = \{ \bm{x}_1, \bm{x}_2, ..., \bm{x}_n \}$ which are assumed to be drawn independently from $p(D\,|\,\bm{\theta}\ )$:
\begin{equation}
  p(D\,|\,\bm{\theta}\ ) = p(\bm{x}_1\,|\,\bm{\theta}\ )\ \cdot \ p(\bm{x}_2\,|\,\boldsymbol{\theta}\ ) \ \cdot \ ...\ p(\bm{x}_n\,|\,\bm{\theta}\ )
  = \prod_{k = 1}^n p(\bm{x}_k\,|\,\bm{\theta}\ )
\end{equation} 

\noindent Maximum likelihood seeks to find the optimum values for the parameters by
maximizing a likelihood function form the training data. The log-likelihood is
given by

\begin{equation}
    l(\theta) = \sum_{k=1}^n \mathit{ln}\ p(x_k|\theta)
\end{equation}

\section{Gaussian distribution [10pt]}

A Gaussian distribution is a type of continuous probability distribution for a
real-valued random variable. Univariate Gaussian distribution that is of the form

\begin{equation}
    p(x) = \frac{1}{\sqrt{2\pi\sigma^2}}\exp{\bigg[ -\frac{1}{2} \bigg( \frac{x-\mu}{\sigma} \bigg)^2 \bigg]}
\end{equation} \\
\newline

\noindent Multivariate Gaussian distribution that is of the form
\begin{equation}
    p(\bm{x}) = \frac{1}{(2\pi)^{d/2} |\Sigma|^{1/2}}\exp{\bigg [-\frac{1}{2}(\bm{x -\mu})^t\Sigma^{-1}(\bm{x-\mu}) \bigg]}
\end{equation}

\newpage
\section{Estimation method derivation [10pt]}
\begin{center}
Please show the estimation results of maximum likelihood for $p(x) \sim N(\mu|\sigma^2)$, where both $\mu$ and $\sigma^2$ are unknown.
\end{center}
\includegraphics[width=\textwidth]{cool_cat}

\end{document} 
